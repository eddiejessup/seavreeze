I'm currently finishing a PhD in computational biophysics in the Physics department of Edinburgh University. Before then I graduated from Durham University, where I studied Theoretical Physics.

My PhD has consisted mainly of writing and running software, written in Python with numerically intensive parts in C, for numerical simulations of the movement of large numbers of bacteria inside water droplets. The analysis of these has involved the visualisation of the positions and orientations of thousands of three-dimensional objects over time. Designing the simulations themselves has involved careful implementation of rotations and displacements of bacteria due to collisions with each other and the environment.

Outside of my work, I've developed many projects from scratch, mostly learning the necessary skills in my spare time. I developed the back-end of a prototype web application whose aim was to generate a London tube-style map showing pubs near a given location. This used Google maps' API to grab the pub locations, which were then connected into a graph and their locations perturbed to form the schematic style of the tube map. This gave me experience of working with graphs, and libraries used for their manipulation (the Python library NetworkX) and visualisation (the Javascript library D3.js).

Through modifying molecular dynamics software for my own research purposes I've experience of dealing with large object-oriented libraries in C++, and through teaching numerous courses I've good experience using Java. I'm very comfortable developing in Mac and Linux environments, and using software tools such as version control and performance profiling.

Over the course of my doctoral work I've realised that the aspects I most enjoy are on the computational, rather than scientific side, and as such have developed an interest in design patterns, algorithms and so on, and see the work I do in the future lying in this area. I'd love an opportunity to join a team where I could use my quantitative background to solve what sounds like an exciting set of challenges and, in the process, make useful things.
